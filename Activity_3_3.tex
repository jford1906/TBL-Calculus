\documentclass{article}
\usepackage[utf8]{inputenc}
\usepackage{fancyhdr}
\usepackage[margin=1.0in]{geometry}
\usepackage{amsmath}
\usepackage{changepage}
\usepackage{graphicx}
\usepackage[inline]{enumitem}
\usepackage{enumitem}
\pagestyle{fancy}
\chead{Math 121 - Related Rates}

\begin{document}
\begin{enumerate}
\item Start by drawing a rectangle (not a square). The length of one side should be three times larger than that of the other side. Even if the side lengths change, that ratio should be constant.
\begin{itemize}
\item Label the sides of the rectangle. Use two different variables.
\item Suppose the short side is decreasing at 2 units per minute. At what rate is the long side decreasing? Start by writing an equation relating the lengths of the two sides, and differentiating with respect to time.
\item Suppose the short side is decreasing at 2 units per minute. At what rate is the area decreasing when the short side is 6 units long? (Hint: How long is the longer side when the short side is 6 units long?)
\end{itemize}
\item Assume a circular outdoor ice rink is melting (and because it's a math problem, it's always a perfect circle as it melts.) If the area is decreasing at $0.5 m/s$, how fast is the radius of the circle changing when the area is 12 $m^2$? 
\item A student sets up a model rocket. She lays out a lead wire, so she is 350m back from the rocket when she fires it. The rocket heads straight up at $15m/s$. How fast is the distance between the student and the rocket changing 20 seconds after is launches? 40 seconds? 1 minute? (Assumptions are that the rocket is of negligible height, and that it will never come down. If you want to make it more realistic, let me know and we can work that part out too.) 
\end{enumerate}

\end{document}