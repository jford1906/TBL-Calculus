\documentclass{article}
\usepackage[utf8]{inputenc}
\usepackage{fancyhdr}
\usepackage[margin=1.0in]{geometry}
\usepackage{amsmath}
\usepackage{changepage,stackrel}
\usepackage{graphicx}
\usepackage[inline]{enumitem}
\usepackage{enumitem}
\pagestyle{fancy}
\chead{Math 121 - Approximating Area Under a Curve}

\begin{document}
\begin{enumerate}
\item Approximate the area under the curve $f(x) = x^2$, over the interval [0,1].
\begin{itemize}
\item Sketch the picture, making it pretty big.
\item Let's set $n=4$. Divide the interval from [0,1] into 4 equal pieces, and label each endpoint on the $x$-axis. Use $x_0 = 0$ and $x_4 = 1$. Use the other subscripts for the points in between.
\item For each $x_i$, label the point on the graph above it. You have the function, and you know the value of each of these endpoints, so you should get the exact $x$ and $y$ values for each point.
\item Use the 4 endpoints on the left to find the height of 4 rectangle. They should all have the same base width. Find and sum up their areas.
\item Now do the same thing, using the 4 endpoints on the right. Do you get a different answer?
\item The actual area is 1/3. How close did you get? How could you get closer?
\end{itemize}
\item Why can't we use this method to find the area under the curve $f(x) = \frac{1}{x}$ for any interval containing 0?
\item Find the area under the curve $f(x) = x^3$ over the interval $[-1,0]$, with $n=6$
\item Without choosing a value for $n$, can you find the exact area under the curve $f(x) = \sin(x)$ over the interval $[-\pi,\pi]$? Try sketching the graph.
\end{enumerate}
\end{document}