\documentclass[12pt,landscape]{article}
\usepackage[margin=1in]{geometry}
\usepackage{graphicx}
\usepackage{multirow}
\usepackage{multicol}
\usepackage{setspace}

\setlength\parindent{0pt}
\usepackage{hyperref}
\usepackage{amssymb, fancyhdr, comment}
\pagestyle{fancy}

\setlength{\columnsep}{2cm}
%\setlength{\columnseprule}{0pt}

\begin{document}
\chead{Math 121 - Learning Objectives - Practice Problems}

\begin{multicols}{2}
\begin{enumerate}
\item Evaluate a limit using the limit laws.\\
\textbf{Textbook:} p. 44, 11-28\\
\textbf{Online:} \url{http://tutorial.math.lamar.edu/Problems/CalcI/ComputingLimits.aspx}
\item Evaluate a limit at infinity.\\
\textbf{Textbook:} p. 68, 19-33\\
\textbf{Online:} \url{http://tutorial.math.lamar.edu/Problems/CalcI/LimitsAtInfinityI.aspx}
\url{}
\item Evaluate an infinite limit.\\
\textbf{Textbook:} p. 68 - 13-18\\
\textbf{Online:}\url{http://tutorial.math.lamar.edu/Problems/CalcI/InfiniteLimits.aspx}
\item Use limits to determine the horizontal and vertical asymptotes of a function.\\
\textbf{Textbook:} p. 68, 34-36, p. 72, 37-38\\
\textbf{Online:} See objective 2
\item Show, via the definition, that a function is continuous at a point.\\
\textbf{Textbook:} p. 54-55, 15-26\\
\textbf{Online:} \url{http://tutorial.math.lamar.edu/Problems/CalcI/Continuity.aspx}
\item Apply the Intermediate Value Theorem to show a function has a zero on a given interval.\\
\textbf{Textbook:} p. 55, 39-42\\
\textbf{Online:} See objective 5
\vspace{1in}
\item Calculate the derivative of a polynomial function directly from the definition.\\
\textbf{Textbook:} p. 81, 3-6, p.82, 25-30, p.93, 19-27\\
\textbf{Online:} \url{http://tutorial.math.lamar.edu/Problems/CalcI/DefnOfDerivative.aspx}
\item Calculate the derivative of a polynomial function using the power rule.\\
\textbf{Textbook:} p. 105, 1-6, 9-22, 25, 26\\
\textbf{Online:} \url{http://tutorial.math.lamar.edu/Problems/CalcI/DiffFormulas.aspx}
\item Calculate the derivative of a trigonometric function.\\
\textbf{Textbook:} p. 105, 7-8, 23-24, 33, 35, p. 112, 29-30\\
\textbf{Online:} \url{http://tutorial.math.lamar.edu/Problems/CalcI/DiffTrigFcns.aspx}
\item Calculate the derivative of a function using the product rule.\\ 
\textbf{Textbook:} p. 112, 2-10, 26\\
\textbf{Online:} \url{http://tutorial.math.lamar.edu/Problems/CalcI/ProductQuotientRule.aspx}
\item Calculate the derivative of a function using the quotient rule.\\
\textbf{Textbook:} p. 112, 11-25\\
\textbf{Online:} See objective 10
\item Calculate the derivative of a function using the chain rule.\\
\textbf{Textbook:} p. 120, 1-16\\
\textbf{Online:} \url{http://tutorial.math.lamar.edu/Problems/CalcI/ChainRule.aspx}
\item Calculate the derivative of a function using a combination of the power, product, quotient, and/or chain rule.\\
\textbf{Textbook:} p. 120, 17-42\\
\textbf{Online:} See objective 12
\item Find the equation of a tangent line to a curve at a given point.\\
\textbf{Textbook:} p. 105, 29-30, p. 112, 27-30, p. 121, 47-48, p. 142, 45-46\\
\textbf{Online:} \url{http://tutorial.math.lamar.edu/Problems/CalcI/DerivativeInterp.aspx}
\item Correctly find the derivative of an implicit function.\\
\textbf{Textbook:} p. 127, 3-16 \\
\textbf{Online:} \url{http://tutorial.math.lamar.edu/Problems/CalcI/ImplicitDiff.aspx}
\item Correctly set up a problem involving at least two related rates.\\
\textbf{Textbook:} p. 132, 1-21\\
\textbf{Online:} \url{http://tutorial.math.lamar.edu/Problems/CalcI/RelatedRates.aspx}
\item Solve a problem involving at least two related rates.\\
\textbf{Textbook:} p. 132, 1-21\\
\textbf{Online:} See objective 16
\item Identify the intervals on which a function is increasing and/or decreasing.\\
\textbf{Textbook:} p. 164, 1-6, 21-32\\
\textbf{Online:} See objective 21
\vspace{1in}
\item Identify the intervals on which a function is concave up and/or concave down.\\
\textbf{Textbook:} p. 164, 1-6, 21-32\\
\textbf{Online:} See objective 22
\item Find all critical values of a function.\\
\textbf{Textbook:} p. 151, 23-34 \\
\textbf{Online:} \url{http://tutorial.math.lamar.edu/Problems/CalcI/CriticalPoints.aspx}
\item Use the 1st derivative test to classify extrema of a function. \\
\textbf{Textbook:} p. 164, 1-6, 21-32 \\
\textbf{Online:}\url{http://tutorial.math.lamar.edu/Problems/CalcI/ShapeofGraphPtI.aspx}
\item Use the 2nd derivative test to classify extrema of a function.\\
\textbf{Textbook:} p. 164, 1-6, 21-32\\
\textbf{Online:} \url{http://tutorial.math.lamar.edu/Problems/CalcI/ShapeofGraphPtII.aspx}
\item Apply the Extreme Value Theorem to a problem.\\
\textbf{Textbook:} p. 151, 35-44\\
\textbf{Online:} \url{http://tutorial.math.lamar.edu/Problems/CalcI/AbsExtrema.aspx}
\item Apply the Mean Value Theorem to a problem.\\
\textbf{Textbook:} p. 157, 9-12\\
\textbf{Online:} \url{http://tutorial.math.lamar.edu/Problems/CalcI/MeanValueTheorem.aspx}
\item Correctly set up an optimization problem using the methods of Calculus.\\
\textbf{Textbook:} p. 180, 1-27\\
\textbf{Online:} \url{http://tutorial.math.lamar.edu/Problems/CalcI/Optimization.aspx}
\item Solve an optimization problem using the methods of Calculus.\\
\textbf{Textbook:} p. 180, 1-27 \\
\textbf{Online:} \url{http://tutorial.math.lamar.edu/Problems/CalcI/MoreOptimization.aspx}
\item Calculate an antiderivative of a polynomial function.\\
\textbf{Textbook:} p. 194, 1-8, 13-14, 17-19, 23-26\\
\textbf{Online:} See objective 32
\item Calculate an antiderivative of a trigonometric function.\\
\textbf{Textbook:} p. 194, 9-11, 20-21, 27, 31\\
\textbf{Online:} See objective 32
\item Use a finite summation to approximate the area under a curve.\\
\textbf{Textbook:} p. 208, 1-8\\
\textbf{Online:} \url{http://tutorial.math.lamar.edu/Problems/CalcI/AreaProblem.aspx}
\item Calculate a definite integral using the Riemann Sum.\\
\textbf{Textbook:} p. 221, 1-6\\
\textbf{Online:} See objective 29
\vspace{1in}
\item Evaluate a definite integral using the Fundamental Theorem of Calculus.\\
\textbf{Textbook:} p. 231, 1-30 \\
\textbf{Online:} \url{http://tutorial.math.lamar.edu/Problems/CalcI/ComputingDefiniteIntegrals.aspx}
\item Evaluate an indefinite integral.\\
\textbf{Textbook:} p. 231, 29-40, 43-48\\
\textbf{Online:} \url{http://tutorial.math.lamar.edu/Problems/CalcI/ComputingIndefiniteIntegrals.aspx}
\item Evaluate an indefinite integral using substitution.\\
\textbf{Textbook:} p. 247, 1-30\\
\textbf{Online:} \url{http://tutorial.math.lamar.edu/Problems/CalcI/SubstitutionRuleIndefinite.aspx}
\item Calculate the derivative of a logarithmic function.\\
\textbf{Textbook:} p. 268, 15-32\\
\textbf{Online:} \url{http://tutorial.math.lamar.edu/Problems/CalcI/DiffExpLogFcns.aspx}
\item Calculate the derivative of an exponential function.\\
\textbf{Textbook:} p. 282, 23-38\\
\textbf{Online:} See objective 34
\item Calculate a derivative using logarithmic differentiation.\\
\textbf{Textbook:} p. 268, 21-24\\
\textbf{Online:} \url{http://tutorial.math.lamar.edu/Problems/CalcI/LogDiff.aspx}
\item Calculate the derivative of an inverse trigonometric function.\\
\textbf{Textbook:} p. 297, 16-23\\
\textbf{Online:} \url{http://tutorial.math.lamar.edu/Problems/CalcI/DiffInvTrigFcns.aspx}
\item Evaluate a limit using L'Hospital's rule.\\
\textbf{Textbook:} p. 311, 1-38\\
\textbf{Online:} \url{http://tutorial.math.lamar.edu/Problems/CalcI/LHospitalsRule.aspx}
\end{enumerate}

\end{document}