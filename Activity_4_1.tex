\documentclass{article}
\usepackage[utf8]{inputenc}
\usepackage{fancyhdr}
\usepackage[margin=1.0in]{geometry}
\usepackage{amsmath}
\usepackage{changepage}
\usepackage{graphicx}
\usepackage[inline]{enumitem}
\usepackage{enumitem}
\pagestyle{fancy}
\chead{Math 121 - The First Derivative Test}

\begin{document}
For each of the functions below, find all of the following:
\begin{itemize}
\item All intervals where the function is increasing.
\item All intervals where the function is decreasing.
\item All local extrema, using the first derivative test.
\item All absolute extrema, when they exist.
\end{itemize}
\begin{enumerate}
\item $f(x) = -x^5  + \frac{5}{2}x^4 + \frac{40
}{3}x^3 + 5$
\item $g(x) = 2x^4 -16x^3 + 20x^2 - 7$
\item $h(x) = \frac{2 - 3x}{x^2 + 1}$
\end{enumerate}

Use the Extreme Value Theorem to find all absolute extrema of the following functions on the given interval. You should be able to find the $x$ values exactly, but the $y$ values may need to be approximated with a calculator.

\begin{enumerate}
\setcounter{enumi}{3}
\item $f(x) = x^3 - 2x^2 -7x$ on the interval $[-2,5]$
\item $h(x) = x^2(10-x)^{2/3}$ on the interval [2, 10.5].
\item $g(x) = \sin(x^2 + 4x + 14)$ on the interval $[-4,-1]$.
\end{enumerate}
\end{document}