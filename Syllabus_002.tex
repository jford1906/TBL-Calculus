\documentclass[11pt]{article}
\usepackage[margin=1in]{geometry}
\usepackage{graphicx}
\usepackage{multirow}
\usepackage{setspace}
\pagestyle{plain}
\setlength\parindent{0pt}
\usepackage{hyperref}
\usepackage{amssymb}


\newcommand{\bC}{{\bf C}}
\newcommand{\bB}{{\bf B}}
\newcommand{\bA}{{\bf A}}
\newcommand{\bD}{{\bf D}}
\newcommand{\bF}{{\bf F}}


\begin{document}

% Course information
\begin{tabular}{ l l }
 \multirow{3}{*}{\includegraphics[height=1.5in,width=1.5in]{GAC_logo.png}} & \LARGE MCS 121 \\\\
  & \LARGE  Calculus I \\\\
  & \LARGE MTHF - 10:30 am - 11:20 am, Olin Hall 317 \\\\
\end{tabular}
\vspace{10mm}

% Professor information
\begin{tabular}{ l l }
  \multirow{6}{*}{\includegraphics[scale=0.05]{Fa16_selfie.jpg}} & \large Jeff Ford \\\\
  & \large jford@gustavus.edu \\
%  & \large website \\
  & \large Office: Olin Hall 307 \\
  & \large Office Hours: Available at drford.youcanbook.me \\
  & \large 507-933-7476 \\
\end{tabular}
\vspace{5mm}
\begin{center} This syllabus is subject to revision. All changes will be posted in Moodle, and will be communicated to students via email. \\
\end{center}

% Course details
\textbf {\large \\ Course Description:} Limits, continuity, the derivative and its applications, and the integral and its applications.  \\

\textbf {\large Text(s):} \emph{Essential Calculus}, 2\textsuperscript{nd} Edition

\textbf {Author(s):} James Stewart;  \textbf {ISBN:} 9781133112761 \\

\textbf {\large Learning Outcomes:} \\
At the completion of this course, students will be able to:
\begin{enumerate} \itemsep-0.4em
  \item Understand the concept of a limit, and compute limits of functions.
  \item Understand the concept of a derivative, and be able to compute the derivatives of various functions.
  \item Apply limits and the derivative to applications in physics and engineering.
  \item Understand the concept of an integral, and be able to compute the integral of various functions.
\end{enumerate}

\section*{Standards Assessments}
At the end of each two-week module, you will have a standards assessment. These are chances for you to demonstrate that you fully grasp a learning objective of the course. Standards will be graded as Correct $(\checkmark)$ or Incorrect $(\times)$. There are 38 learning objectives in the course. All standards covered up to that point in the course will be available on each assessment. You may choose which problems you complete. Each objective must be completed twice, on separate assessments, in order to count as mastered. Certain more complicated problems may count as covering multiple objectives.

% I recommend using \newpage here if necessary

\newpage
\section*{Team-Based Learning}
This course will be using a team-based learning (TBL) approach. TBL encourages self-directed learning and will help teach you how to apply what you learn in a collaborative environment. TBL requires you to be prepared for and attend classes. Using the TBL method will allow us to avoid long lectures so that we can dig into the more complicated business of critical thinking about what we are learning. The course components will include the following:
\begin{itemize}
\item \textbf{Readiness Assessments (RA)}: At the beginning of classes where there is an assigned reading due, we will have a closed-note Readiness Assessment (10 items). You will take this RA twice; you will complete it once as an individual (iRA), and then as a team (tRA). These quizzes will be scored immediately so you can appeal your scores on the RA if you believe the key is wrong or there is another correct answer listed. If you are absent, you will receive a zero for both the iRA and the tRA for that day; however, I will also drop your lowest iRA and tRA at the end of the year, so you can miss one without penalty. I do not allow makeups for these assessments. 

\item \textbf{Team Activities:} In most classes, we will have one or more application activities where you will need to work with your team to devise the best solution or approach to a selection problem. You will not need to work on these assignments outside of the classroom, although completing the required readings will be imperative for your success on these assignments. The grading for this will assess whether you have appropriately applied key concepts you’ve learned to the problem. You will sign in to a sheet with your team each day to indicate that you've completed the Activity. Each daily activity is worth 3 points.

\item \textbf{Peer Reviews:} Twice during the course, you will complete a peer review. These will be anonymous, but they will be shared with your teammates. The goal of this review is to give you an opportunity to provide constructive feedback to your team members. Your average rating on the second review will serve as a multiplier on your team performance score. If you do not complete the reviews, you receive a score of zero for your own review, which will lead to a failing grade in the course.
\end{itemize}

\section*{Grading}
\begin{center}
\begin{tabular}{|c|c|c|c|}
\hline
Desired Grade&Objectives Mastered&Desired Grade&Objectives Mastered\\
\hline
\textbf{A}&35&\textbf{C}&27\\
\hline
\textbf{A-}&34&\textbf{C-}&26\\
\hline
\textbf{B+}&33&\textbf{D+}&25\\
\hline
\textbf{B}&31&\textbf{D}&23\\
\hline
\textbf{B-}&30&\textbf{D-}&22\\
\hline
\textbf{C+}&29&\textbf{F}&$\leq 22$\\
\hline
\end{tabular}\\
There are a total of 196 points available for Readiness Assessments and Team Activities. Grades are capped at the team score percentage (90\% for an \textbf{A}, 80\% for a \textbf{B}, etc.). A student whose team grade exceeds their standards grade, will have their grade raised by 1/3. For example, if a student mastered 33 objectives, but scored over 90\% on the team grade, that student would receive an $\textbf{A-}$. On the other hand, if a student mastered 35 objectives, but only scored 70\% on the team grade, that student would receive a \textbf{C}.
\noindent

\end{center}
\newpage 

\textbf {\large Incomplete Grades:}
		\begin{itemize}
			\item  In general, a grade of “incomplete” will only be given if a student has a documented reason for missing a substantial portion of the class (i.e., prolonged illness, death in 			immediate family) AND he/she was passing the course. \\
		\end{itemize}
\textbf {\large Academic Honesty}
		\begin{itemize}
			\item Students are expected to work independently. \textbf{Offering} and \textbf{accepting} solutions from others is an act of \textbf{plagiarism}, which is a serious offense and 			\textbf{all involved parties will be penalized according to the Academic Honesty Policy}. Discussion among students is encouraged, but when in doubt, direct your questions 					to the instructor. Full descriptions of the Academic Honesty Policy and the Honor Code can be found in the Academic Catalog (online at 		 																\url{https://gustavus.edu/general_catalog/current/acainfo}). 
		\end{itemize}	

\textbf{\large Research Help:}

\hspace{3mm}
\hangindent=5mm You can always get help with your research at the library. Reference librarians will help you find information on a topic, develop search strategies for papers and projects, search library catalogs and databases, and provide assistance at every step. Drop-ins and appointments are both welcome. Visit \url{https://gustavus.edu/library/reference_question.php} for hours, location, and more information.\\

\textbf {\large Disability Services:}

\hspace{3mm}
\hangindent=5mm Gustavus Adolphus College is committed to ensuring the full participation of all students in its programs. If you have a documented disability (or you think you may have a disability of any nature) and, as a result, need reasonable academic accommodation to participate in class, take tests or benefit from the College’s services, then you should speak with the Disability Services staff, for a confidential discussion of your needs and appropriate plans. Course requirements cannot be waived, but reasonable accommodations may be provided based on disability documentation and course outcomes. Accommodations cannot be made retroactively; therefore, to maximize your academic success at Gustavus, please contact Disability Services as early as possible. Disability Services (\url{https://gustavus.edu/advising/disability/}) is located in the Academic Support Center. Disability Services Coordinator, Kelly Karstad, (kkarstad@gustavus.edu or x7138), can provide further information.\\

\newpage

\textbf{\large Help for Multilingual Students:}

\hspace{3mm}
\hangindent=5mm Support for English learners and multilingual students is available through the Academic Support Center’s Multilingual Learner Tutor (\url{https://gustavus.edu/advising/}). The MLL tutor can meet individually with students for tutoring in writing, consulting about academic tasks, and helping students connect with the College’s support systems. When requested, the MLL tutor can consult with faculty regarding effective classroom strategies for English learners and multilingual students. If requested, the MLL tutor can provide students with a letter to a professor that explains and supports appropriate academic arrangements (e.g., additional time on tests, additional revisions for papers). Professors make decisions based on those recommendations at their own discretion. In addition, English learners and multilingual students can seek help from peer tutors in the Writing Center (\url{https://gustavus.edu/writingcenter/}).\\

\textbf{\large Title IX:}

\hspace{3mm}
\hangindent=5mm Title IX is federal legislation that makes clear that violence and harassment based on sex or gender are civil rights violations. Gustavus Adolphus College takes incidents of sexual misconduct and sexual harassment seriously.
 
\hangindent=5mm Sexual misconduct includes the following: non-consensual sexual contact, non-consensual sexual intercourse, sexual exploitation (taking non-consensual or abusive sexual advantage of another), intimate partner violence (physical, sexual, or psychological harm by a current or former partner or spouse), and stalking. (Please see the Student Sexual Misconduct Policy in the Gustavus Guide for more details and definitions or online at:\url{ https://gustavus.edu/deanofstudents/policies/gustieguide/sexual-assault.php#misconduct}).
 
\hangindent=5mm Sexual Harassment is any behavior of a sexual nature that is unwelcome, offensive or fails to respect the rights and dignity of another person whether of the same or opposite sex (please see the All-College Policy against Harassment and Sexual Harassment for examples and more details: \url{https://gustavus.edu/facultybook/allcollegepolicies/#Anchor-Sexua-60443}).\\

Please email the instructor a picture of your favorite spaceship to confirm that you have actually read the entire syllabus.

\end{document}
