\documentclass{exam}

\usepackage{amsmath,amssymb,amsthm}

\title{Math 121 - Readiness Assessment \# 5}
\date{}
%Key - D029

\begin{document}
\maketitle

\begin{questions}
\question
If I claim that the function $g(x)$ is an antiderivative of the function $f(x)$, this means:
\begin{choices}
\choice $f'(x) = g(x)$
\choice $f'(x) = g'(x)$
\choice $g'(x) = f(x)$
\choice $f(x) = g(x)$
\end{choices}
\question
We say \textit{an} antiderivative, rather than \textit{the} antiderivative because:
\begin{choices}
\choice the derivative of a constant is always zero.
\choice there might be others we haven't thought of.
\choice we have to use the limit definition to find it.
\choice the voices in my head said so.
\end{choices}
\question
The \textit{area problem} is a very old problem, interested in doing what?
\begin{choices}
\choice Showing all of the ways we can calculate area.
\choice Approximating areas with rectangles.
\choice Finding the exact area under a curve, over a given interval.
\choice Showing that some areas are infinite.
\end{choices}
\question
When using sigma notation, the symbols $$\sum_{i=1}^n f(x_i) \Delta x$$ means what?
\begin{choices}
\choice $f(x_1) + f(x_n)$
\choice $f(x_1)\Delta x + f(x_2)\Delta x + \cdots + f(x_n)\Delta x$
\choice $f(x_1)\Delta x + f(x_n)\Delta x $
\choice $f(x_1)+ f(x_2) + \cdots + f(x_n)$
\end{choices}
\question
The definition of area in the textbook requires how many approximating rectangles?\begin{choices}
\choice Infinitely many, so we have to use a limit.
\choice $n$, where $n$ is as some fixed number.
\choice Infinitely many, so we can't do it precisely.
\choice It depends on the type of area you're finding.
\end{choices}
\newpage
\question
Which of these is the notation for the definite integral?
\begin{choices}
\choice $\int_a^b f(x) dx$
\choice $\int f(x)$
\choice $\sum_{i=1}^n f(x_i) \Delta x$
\choice $\int_a^b f(x)$
\end{choices}
\question
A theorem in the textbook tells us that, for a function to be integrable on an interval $[a,b]$, it must be
\begin{choices}
\choice Differentiable everywhere
\choice Approximated using a finite number of rectangles
\choice Continuous everywhere
\choice Continuous, except at a finite number of points
\end{choices}
\question
If we had to evaluate a definite integral by it's definition, what would we do?\begin{choices}
\choice $\sum_{i=1}^n f(x_i) \Delta x$
\choice $\sum_{i=1}^n f(i)$
\choice $\lim_{n\rightarrow \infty} \sum_{i=1}^n f(x_i) \Delta x$
\choice $\lim_{n\rightarrow \infty} \sum_{i=1}^n f(i)$
\end{choices}
\question
The second part of the fundamental theorem of calculus allows you to do what?
\begin{choices}
\choice Differentiate an integral to get a new function.
\choice Find velocity if you already know acceleration.
\choice Plug in the limits of integration into your original integrand.
\choice Evaluate a definite integral by finding the value of an antiderivative at the upper and lower limits of integration.
\end{choices}
\question
How is an indefinite integral evaluated?
\begin{choices}
\choice Taking the derivative of the integrand.
\choice Finding an antiderivative, and not forgetting the $+C$ on the end.
\choice Using the first part of the fundamental theorem of calculus.
\choice Using the second part of the fundamental theorem of calculus.
\end{choices}

\end{questions}
\end{document}