\documentclass[11pt]{article}
\usepackage[margin=1in]{geometry}
\usepackage{graphicx}
\usepackage{multirow}
\usepackage{multicol}
\usepackage{setspace}

\setlength\parindent{0pt}
\usepackage{hyperref}
\usepackage{amssymb,amsthm,amsmath, fancyhdr}
\pagestyle{fancy}

\usepackage{lipsum}

\begin{document}
\chead{Math 121 - Trigonometry Review}
\begin{enumerate}
\item Draw the Unit Circle. Include the $x$ and $y$ axes.
\item Choose some angle $\theta$ (in radians), and draw a line from the center of the circle to the edge, making an angle $\theta$ with the $x-$axis.
\item What is the definition of $\sin(\theta)$ and $\cos(\theta)$? Note that I'm not asking for the exact value for the $\theta$ you chose. I want to know how the functions are defined.
\item Define the rest of the trig functions for $\theta$.
\item Draw a right triangle, whose hypotenuse is the line you added to the unit circle in your last drawing. Using the definitions you came up with earlier, explain the SOH-CAH-TOA relationship between the trig functions and the sides of a right triangle. 
\item Explain why the following identities are true for any choice of $\theta$.
\begin{itemize}
\item $\sin^2(\theta) + \cos^2(\theta) = 1$.
\item $1 + \cot^2(\theta) = \csc^2(\theta)$.
\item $\tan^2(\theta) + 1 = \sec^2(\theta)$.
\end{itemize}
\item Remind the instructor to show you the trick for finding the values of common trig functions on reference angles.
\end{enumerate}
\end{document}